Why license:
- general history of formal math libraries with current overview
- short history of Mizar/MML, its current (non)license

Discussion of related licensing models:

- Coq/CoRN - GPL , Gonthier?
- Isabelle/AFP - BSD/LGPL
- Wikipedia, PlanetMath - CC-BY-SA
- HOL Light/ Flyspecl ?
- Arxive - multiple choice between ...?
- Linux - GPL 2 ?


What is formal math library?:
- code (why?) - Coq more code, Mizar less ?
- what is the "code behavior": Curry-Howard? (relevant for Mizar?) Algorithm extraction - Isabelle, etc.
- text (why?) - wikipedia/arxive like, math books
- math (why?) - any difference betwen math text and normal text ("math is not patentable"?)
- finally, "formal math" - category per se?

Linking/adaptation in formal math library:
- is using theorems from other article as GPL linking? (we decided so) why? why not?
- is it like CC-BY-SA adaptation? why (not)?

Why open-source (viral) license?
- virality is fair, promotes public contribution?
- why dual-license? (pros, cons)

Why keep the copyright ownership with SUM?:
- the licenses don't seem optimal now
- the Wikipedia relicensing case
- the NICTA/Isabelle L4 commercial case

What are reasonable conditions for copyright ownership?:
- the FSFE license agreement: - only the viral case
- our modification - we want to take care of the possible future commercial cases (like NICTA L4)
- what are the risks? - fundamentalists will not like the commercial option?
- is it going to work (will people assign copyright?)
