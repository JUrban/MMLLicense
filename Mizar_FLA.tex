%% $Id: FLA.ccasa.en.tex,v 1.1 2007-04-16 11:11:03 shanecoughlan Exp $
%% 
%% Fiduciary License Agreement (Copyright Assignment)
%% of the Free Software Foundation Europe paying respect to the two
%% different Copyright traditions  
%%
%% Language: English
%%

%{{{ Header

\newcommand{\fiduciary}{FSFE}
\newcommand{\fiduciaryaddress}{\begin{alltt}
Free Software Foundation Europe
Talstrasse 110
40217 Dusseldorf
Germany
\end{alltt}}
\documentclass[a4wide,12pt]{article}
% \usepackage{german,umlaut}
\usepackage{fancyheadings}
\usepackage{alltt}
\usepackage{epsfig}
\usepackage[english]{babel}

\selectlanguage{english}
\pagestyle{fancyplain}

\lhead[\fancyplain{}{\bfseries\thepage}]
        {\fancyplain{}{\bfseries\rightmark}}
\rhead[\fancyplain{}{\bfseries\leftmark}]
        {\fancyplain{}{\bfseries\thepage}}
\cfoot{}
\renewcommand{\sectionmark}[1]{\markright{\thesection\ #1}}
\renewcommand{\thesection}{\S~\arabic{section}}

%
% alternative ways of doing it...
%

% this makes the references work correctly:
%\newcommand{\A}{\subsection{}}
%\renewcommand{\subsectionmark}[1]{\markright{\thesection\ }}

% this looks better:
\newcounter{absatz}[section]
\newcommand{\A}{\par\vspace{1ex}
                \stepcounter{absatz}\noindent(\arabic{absatz})~~}

% make the page a little wider
%\addtolength{\textwidth}{3cm}
%\addtolength{\headwidth}{3cm}
%\addtolength{\hoffset}{-1.5cm}
%\addtolength{\textheight}{2cm}
\addtolength{\textwidth}{3cm}
\addtolength{\headwidth}{3cm}
\addtolength{\hoffset}{-1.5cm}
\addtolength{\textheight}{4cm}

%}}}

\begin{document}

%{{{ Head of front page

\thispagestyle{empty}

\begin{center}
{\LARGE\bf Fiduciary Licence Agreement}\\
(Version 1.2)
\end{center}

\begin{center}
\sc\small Copyright (C) 2002-2007 FSFE, e.V.,\\
\sc\small Talstrasse 110,
40217 Dusseldorf,
Germany\\
\sc\small This licence is released under the terms of\\
the Creative Commons Attribution/Share-alike licence version 2.5.
\end{center}

%}}}

\begin{center}
{\Large\bf Preamble}
\end{center}
 
Access to software determines participation in a digital society. To
secure equal participation in the information age, the Free Software
Foundation Europe (FSFE) pursues and is dedicated to the furthering of
Free Software, defined by the freedoms to use, study, modify and
copy. Independent of the issue of commercial exploitation, it is
proprietary, freedom-diminishing licensing that works against the
interests of people and society at large, which is therefore rejected
by FSFE.\\

The purpose of this agreement is to ensure the lasting protection of
Free Software by making FSFE the fiduciary of the author's interests.
It empowers FSFE -- and its sister organisations -- to uphold the
interests of Free Software authors and protect them in court, if
necessary.\\

FSFE is given the right to relicense the software as necessary for the
long-term legal maintainability and protection of the software. The
agreement also grants the author an unlimited amount of non-exclusive
licences by FSFE, which allow using and distributing the program in
other projects and under other licences. \\

The contracting parties sign the following agreement in full
consciousness that by the grant of exclusive licence to the Free
Software Foundation Europe e.V. and by the administration of these
rights the FSFE becomes trustee of the author's interests for
the benefit of Free Software.\\ 

\pagebreak
\begin{center}
{\Large\bf Agreement}
\end{center}

Between 
{\bf (please cross out unsuitable alternatives)}\\
\begin{itemize}
\item the author
\begin{alltt}
    _______________________________________________
    _______________________________________________ (Occupation, D.o. Birth)
    _______________________________________________ (Street)
    _______________________________________________ (ZIP, City, Country)
    _______________________________________________ (Pseudonym, Email)
\end{alltt}
\item the owner of the exclusive licence,
\begin{alltt}
    _______________________________________________ (Company or name)
    _______________________________________________ (HQ or addr.)
    _______________________________________________ (managing director)
\end{alltt}
\begin{itemize}
\item acquired by virtue of a contract date as of
\item contracting party:
\item acquired as employer in the context of a work and service
  relationship
\end{itemize}
\end{itemize}
\begin{flushright}
- hereinafter referred to as "Beneficiary" --
\end{flushright}
and the
\begin{alltt}
    Free Software Foundation Europe, e.V.,
    Talstrasse 110
    40217 Dsseldorf
    Germany
\end{alltt}
\begin{flushright}
- hereinafter referred to as "FSFE" --
\end{flushright}
the following agreement is entered into:\\

\section{Grant}
\A Subject to the provision of {\S}~2, Beneficiary assigns to FSFE the
Copyright in computer programs and other copyrightable material
world-wide, or in countries where such an assignment is not
possible,\footnote{Countries where assignments of the copyright in a
work are impossible include, but are not limited to, Germany, Austria,
Slovenia and Hungary.} grants an exclusive licence, including, inter
alia:
\begin{enumerate}
\item the right to reproduce in original or modified form;
\item the right to redistribute in original or modified form;
\item the right of making available in data networks, in particular
  via the Internet, as well as by providing downloads, in original or
  modified form;
\item the right to authorize third parties to make derivative works of
  the Software, or to work on and commit changes or perform this
  conduct themselves.
\end{enumerate}
\A Beneficiary's moral or personal rights remain unaffected by this
Agreement.\\ 
\A In some countries, the law may provide that the employer is deemed
to be the owner of the rights on materials developed by an employee in
the course of his or her employment, unless the parties have agreed
otherwise. The Beneficiary is aware of these provisions, and therefore
warrants, represents and guarantees that the Subject Matter is free of
any of his or her employer's exclusive exploitation rights.

\section{Subject Matter}

The rights and licences granted in {\S}~1 are subject to all
``Software'' and ``Documentation.'' For the purpose of this Agreement,
``Software'' shall mean all computer programs, copyrightable sections
of computer programs or modifications of computer programs that have
been developed or programmed by Beneficiary and that are specified in
this Agreement below or that are listed in Exhibit A attached to this
Agreement and dated and signed by the contracting parties. Likewise,
``Documentation'' shall refer to all manuals and documentation written
by Beneficiary alongside and usually distributed with the ``Software''
and are similarly specified below or listed in Exhibit A:
\begin{alltt}
    __________________________________________________________________
    __________________________________________________________________
    __________________________________________________________________
\end{alltt}
Except in countries where such an assignment is not
possible,\footnote{Countries where assignments of the copyright in a
future work are impossible include, but are not limited to, France.}
the rights [and licences] granted under this agreement by Beneficiary
shall also include future developments, future corrections of errors
or faults and other future modifications and derivative works of the
software that Beneficiary obtains copyright ownership.  Excluded from
this provision are modifications that are not derived from the subject
matter and that have to be regarded as independent and original
software. 

\section{FSFE's Rights and \\Re-Transfer of Non-Exclusive Licence}
\A FSFE shall exercise the granted rights and licences in its
own name. Furthermore, FSFE shall be authorized to enjoin third
parties from using the software and forbid any unlawful or copyright
infringing use of the Software, and shall be entitled to enforce all
its rights in its own name in and out of court. FSFE shall also
be authorized to permit third parties to exercise FSFE's rights
in and out of court.\\ 

\A FSFE grants to Beneficiary a non-exclusive, worldwide, perpetual
and unrestricted licence in the Software. This right's [and licence's]
scope shall encompass and include all the rights [and licences]
specified in {\S}~1. Furthermore, FSFE grants to Beneficiary
additional non-exclusive, transferable license to use, reproduce,
redistribute and make available the Software as needed for releases of
the Software under other licences.  This re-transfer shall not limit
the scope of FSFE's exclusive licence in the Software and FSFE's
rights pursuant to {\S}~1.\\

\A FSFE shall only exercise the granted rights and licences in
accordance with the principles of Free Software as defined by 
the Free Software Foundations. FSFE guarantees to use the rights 
and licences transferred in strict accordance with the
regulations imposed by Free Software licences, including, but not
limited to, the GNU General Public Licence (GPL) or the GNU Lesser
General Public Licence (LGPL) respectively. In the event FSFE violates
the principles of Free Software, all granted rights and licences shall
automatically return to the Beneficiary and the licences granted
hereunder shall be terminated and expire.\\

\A The transfer of the rights and licences specified in {\S}~1 shall
be unrestricted in territory and thus shall apply world-wide and be
temporally unlimited.

\section{Miscellaneous}
\A Regarding the succession of rights in this contractual
relationship, German law shall apply, unless this Agreement imposes
deviating regulations. In case of the Beneficiary's death, the
assignment of exclusive rights shall continue with the heirs. In case
of more than one heir, all heirs have to exercise their rights through
a common authorized person.\\

\A Place of jurisdiction for all legal conflicts arising out of or in
connection with this Agreement is Munich, Germany.\\

\enlargethispage{2cm}
\vspace{3cm}
\begin{alltt}
__________________, ______________    __________________, ______________




__________________________________    __________________________________
           (Beneficiary)                         (FSFE)
\end{alltt}

\end{document}
